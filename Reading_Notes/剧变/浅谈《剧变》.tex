\documentclass{article}
\usepackage{ctex}

\title{浅谈《剧变》}
\author{刘铭}
\date{2020/7/4}

\begin{document}
    
    \maketitle
    \setcounter{page}{0}
    \thispagestyle{empty}

    \newpage

    \section{初识}
    但我第一次听到这本书的时候是在b站的一个up主口中,得知这本这么nb的书籍,在那位up口中,这本书简直是无可挑剔,在世预言家,将前两个月以及现在美国的局势描述和预测的非常准确。
    
    因此,我对这本书的兴趣大增,在第二天果断上豆瓣买了这本书,于是两三天后拿到了这本书。刚开始拿到这本书的心情还是很好的,翻阅目录,整本书大致分为三个板块,第一板块讲述作者的危机理论,通过作者对自己一生的概括,引出自己对个人危机以及国家危机的理解,以及对危机所建立的危机框架;第二板块讲述了作者所研究的6个国家在历史上所经历的危机,以及这些国家如何解决这些危机,最后将这些国家套用作者的危机理论,验证危机理论的正确性或者重要性;第三板块讲述的是作者认为当今世界日本和美国以及世界所可能面临的危机,使用他的危机模型来讲述这些国家或者世界应该何取何从!

    在翻阅完目录以及大概翻了翻书中的内容后,我感觉整本书主要是讲述老先生的这套危机理论,分别从自己身上,过去国家身上以及未来或者现在国家所经历的危机中,来论述老爷子的这套危机理论,但是单从模型上来说,个人感觉缺乏数据来验证,虽然作者在开篇就提到,这一系列的研究,所收集的资料累积起来比他的身高还高,用一本书来写不好应用太多数据,但是书中如果有模型图,或者其他可视化的数据模型来一同论述,我认为会更加好的了解作者的危机模型。可能术业有专攻,不同学科门类的差别太大,也许是社科类的研究使用可视化的模型不太常用或者非常困难,这也是我的一家之言。

    最后说回初识这本书的“买家秀”和当时从up主口中所描述的“卖家秀”有什么区别或者不同。当时up主推荐所引用书中的话语有很多一部分是从作者的理论模型摘取的,以及作者对美国危机的预测。这两部分前者是在书的内容最前面部分,而后者是在书的最后部分。而当时我翻阅目录时,就直接翻到了当时所感兴趣的内容,也就是书的最后部分讲述美国当前面临的危机部分。由于还没有学习到作者所讲述的危机理论模型,直接看后面作者所写的内容感觉可读性没有那么好,也是最初对这本书的一些误解。其实,整本书所描述,举证的例子都是围绕这作者所提出的危机理论,也就是那12条危机框架。

    \newpage

    \section{细阅}
    在仔细阅读的过程中,慢慢懂得了作者的危机框架,以及作者写作的手法。作者在描写每个国家的危机时,都是从在这个危机发生前一段时间,来讲述当时这个国家处于一个什么状态。然后谈到作者在这个国家的亲身经历,在走访这个国家时询问当地人民,对国家当时的危机有什么体验或者感受,然后谈论这个国家如何解决这个危机,通过什么方式解决这个危机,危机解决后这个国家与过往有什么不同。最后,将这个国家的危机前兆,解决危机的方法以及危机过后国家做出了什么样的调整对应作者的危机框架,从而显示出作者危机框架的有效性。当然,不是所有国家的危机都能够完全对应得上作者得为佳框架,如果那样得话,国家的危机就太容易被发掘或者解决了。通常在12条危机理论中,每个国家能用7-8条危机理论描述面对的危机。

    在阅读的过程中,一方面学到了作者的危机框架,更多的是,了解了每个国家的兴衰史。在这之前,我真不知道芬兰的历史有那么的波折,在我的认知中,芬兰是一个非常发达的国家,在欧洲的北部,它的国家有很高效的领导方式,有非常多的高端企业或者互联网企业在芬兰,而且芬兰的企业管理方式通常是扁平化管理,人们生产生活水平都很高。但是通过阅读这本书中作者讲述的芬兰历史,才了解到芬兰原来它的地缘政治是这么的“卑微”,历史上也因为这样的地缘政治所遭受了这么严重的国家危机历史。可能,那样的地缘政治也成就了芬兰人那种对事物处理的方式方法吧!

    另外了解到的一个历史国家就是智利,我的印象中,智利的足球水平很高,是一个在南美洲的国家,貌似以前对智利的了解也就这么多。但是,通过这本书的阅读,知道了智利原来经历过这么长久的军政府统治时期,还有,他们在历史上也被共产主义的思想沐浴过,但是处于离美国这么近的位置上,共产主义政权被美国干涉内政而不能很好的执政,最终被军政府打压统治了十几年。还有一点,智利以前的民主社会我也是没想到的,与其说没想到,不如说是不敢相信,我认为智利在南美洲可能与巴西,墨西哥等国家差不多,没有什么稳定的政权,国家政权摇摆不定。但是,作为一个民主国家,就因为地缘政治的原因,让民众在军政府的统治下,生活这么多年。可见,当时社会是多么的可怕!

    再一个就是印度尼西亚,在读这本书前不久,我看到一个讲述1965年印尼屠杀中国人的纪录片,里面很多中国人被当时民粹主义的印尼人残忍的杀害。当时我认为印尼在东南亚这个地理位置,应该和越南、印度等国家一样,被西方帝国主义殖民统治,然后二战后通过政治交接,有了自己国家的政权。但是,通过作者从荷兰远航时代讲起,说到当时印尼那个地方连国家都没有,只存在部落,我就很迷!然后他们对于西方列强的到来,并不是像我们一样,联合起来共同抵御外敌,而是像通过外敌的手,将同处现在印尼土地上的其他部落消灭。这是何等的原始社会啊!最后,虽然在西方列强的统治之后,印尼建立起了统一的国家,但是在民族、语言、宗教、信仰上,只有现在的印度能和它有得一拼!

    最后介绍的历史上危机的国家就是现在和我们国家关系很不好的澳大利亚,虽然我对澳大利亚的历史有些了解,知道澳大利亚是以前英国为了惩罚罪犯,而不断移民产生的国家。但是对于“白澳”政策,以前不是很了解。通过这本书的描述,知道了“白澳”政策的来源,但是,作为不管是历史上对英国的归顺,还是现在对美国的奉承,澳大利亚对于自己的国家在我看来,总是没有一个很有气概的定位。作为一个这么好的地理条件和经济条件的国家,没有显示出一个国家该有的气魄与能力,而是每天跟在一个有“奶”的“娘”后面,这是因为人口组成上是“流放”民族的原因,还是一直处于舒适区,没有经历过外界大灾大难的原因?

    这本书最精彩的地方也就是接下来作者所写的第三部分,作者把标题取为“酝酿中的危机”,作者分三个章写的这一部分,第一章讲述了日本现在所面临的困难与挑战,与别人的角度不同的是,作者从自己在日本所经历和周围的日本朋友的角度写了很大一部分对日本的看法,而不是其他人的论述,单从日本的宏观角度来谈论,虽然最终的结果或者解决方案都差不多。第二章也是我认为这本书最值得看的一章——美国的未来,作者对美国的担忧从四个方面讲述,分别是政治极化、选举的僵化、不平等问题以及对未来的投资。如果是在两个月的美国与其描述的现象对比,是不是很佩服戴蒙德老先生的远见。忘了说一句,这本书是去年出版的,也就是老先生至少在19年就感觉美国会经历现如今美国经历的困难。这是多么令人佩服!最后一章,讲述了世界未来可能面临的四个危机,第一是核威胁;第二是全球气候危机;第三是资源危机;第四是全球贫富差距。这四个危机在我看来并不陌生,但是老先生是按照可能发生的顺序来将这四个危机排序的,将核威胁排在第一位是我没有想到的。我的认识中,大国之间核武器通过博弈论,并不太可能出现真正的核战争,但是作者从一些拥有核武器的核“小”国的角度来说,核战争很可能因为某些差错,而爆发!同时,作者推荐了一本美国前国防部长佩里写的《我在核战争边缘的历程》,引发了我一些兴趣。

    \newpage

    \section{启发}
    首先,对老先生这套危机理论的学习使我受益匪浅。也就是老先生认为面对危机时,可能影响危机结果的12个因素,分别是:
    \begin{itemize}
        \item 对国家陷入危机的举国认识
        \item 愿意承担责任
        \item 划清界限/选择性变革
        \item 从他国获取物质和资金方面的帮助
        \item 借鉴他国应对危机的经验
        \item 国家认同
        \item 诚实的国家自我评估
        \item 应对过往国家危机的经验
        \item 应对国家失败的耐心
        \item 特定情况下国家的灵活性
        \item 国家核心价值观
        \item 不受地缘政治约束
    \end{itemize}
    
    最初,作者是从自身经历的危机时刻描述这个危机框架的,也就是这个危机框架最初是适用于个人面对危机时所应该考虑的方面。因此, 在我未来面临或者可能面临危机时,可以尝试采用这套理论来对应当时的自己,看看能否通过从这些理论方法中,将自己从危机中脱离出来或者让自己更好。其次,在以后我看待国家危机或者谈论国际问题时,可以从这几个方面去谈论国家问题,这样显得更加有理论性,或者说更加专业。最后,也是我读完这本书给我感受最深的一点,也就是作者不管在日常生活中,还是在旅行度假中,都很善于了解一个国家历史上所经历的事情,给我感受最深的是作者论述芬兰时,讲到他亲历芬兰所看到的历史博物馆以及与当地人交谈,所了解到的历史与民族情怀。

    如果能做到这一点,再去旅行时,就能做到与老年观光客有所区别了,可以不再仅仅是“上车睡觉,下车拍照”,而能够做到“上车侃侃历史,下车谈谈人文”。那样的旅行或许能够带来更加不一样的体验。

\end{document}